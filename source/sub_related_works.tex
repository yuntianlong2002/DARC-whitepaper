\documentclass[main.tex]{subfiles}
\begin{document}

\section{加密世界中的监管与自我监管}

大多数部署在以太坊、Solana等区块链上的DAO并不受任何政府或法律的监督。智能合约将简单地按照其编程执行。与其受到正式流程和机构的监督,DAOs是由部署智能合约的实体自我调节的。值得注意的是,他们可以通过投票决定更改规则或更新DAO的智能合约。每个DAO都可以像公司、基金或非营利社区一样运作。代币持有者将通过投票决定如何运行DAO。DAO的概念是由Christoph Jentzsch发明的\cite{dao_paper},他是DAO概念的创造者和第一篇DAO论文的作者。尽管最初存在争议,但DAO概念现在在加密世界中非常流行并被广泛使用,导致以太坊社区分裂为当前的“新”以太坊和原始未分叉的“以太坊经典”。

尽管DAO是自我调节的,但它们仍然可以被政府监管,许多政府正试图通过为DAO提供立法服务来监管加密世界,将它们置于政府和税法之下。这允许DAO在注册公司的管理下运行,并从事提供服务、支付和提供产品等商业活动。然而,这些受政府监管的DAO没有完全自我调节DAO那样的自由度。政府监管的DAO不是建立在去中心化基础设施之上的,政府仍然可以通过税法和股市法律来监管它们。虽然这些受监管的DAO可能不像自我调节的DAO那样受欢迎,但政府正在积极探索各种方式来监管加密世界。

另一种运行DAO的方式是不涉及任何法律实体,而大多数DAO今天都是以这种方式运作的。所有操作都由社区代币拥有者提议并投票决定。拥有者可以从去中心化交易所(DEX)如Uniswap和中心化交易所(CEX)如Coinbase、Binance和FTX购买代币,并参与投票过程以决定如何运行DAO。这些DAO的核心贡献者和管理者在智能合约中定义DAO的规则,主要使用Solidity用于以太坊和Rust用于Solana。他们决定它是开源的还是私有的,定义谁可以发起提案,并设置投票规则,如提案持续时间、投票批准门槛和用新代币铸造的投票奖励。

\subsection{当前流行DAO的设计}

DAO的设计随着时间的推移而演变,变得更加复杂并引入了额外的功能。随着在旧DAO中发现的缺陷、漏洞和效率低下问题,当前流行的DAO作为这些问题的解决方案而出现。一些今天流行的DAO包括Aragon、Moloch和DAOStack等。

在设计得当的DAO中,一个显著的特征是存在完全功能的用户界面用于DAO治理。这些努力旨在吸引那些不熟悉区块链编码或编码知识有限的个人,通过用户友好的界面实现。这些DAO不需要太多的技术专长就可以参与,而不牺牲简洁性的任何功能。另一个共同特点是在DAO治理的基本方面实施额外限制:投票。例如,Moloch采用加权投票而不是每个代币一票,而Aragon利用模块化的“法庭”作为争议解决机制,以防止51%攻击和恶意提案。大多数流行的DAO也以模块化或分隔的方式构建,从而受益于增强的组织和可扩展性,这是许多DAO的关键焦点。例如,Aragon包括各种组,如所有代币持有者的组、执行子DAO、合规子DAO和技术组。

虽然这些DAO共享几个特征,但它们通过其独特的关键特征而有所区别。Moloch在多个方面脱颖而出。最初,它采用了许可成员制,只允许现有成员投票决定是否添加新成员。Moloch还引入了“ragequit”功能,使成员能够将所有会员股份换成财务资产。此外,提案在有资格投票前需要赞助,且接受的提案执行前有宽限期。DAO内的权力不可转让并不是一个独特的方面。在Moloch v2中,引入了几个新功能,这些功能显著塑造了当前DAO的格局。这些功能包括支持多类代币,允许使用多达200种代币而不仅仅是一种,资金提案,对非成员开放提交,一个“GuildKick”机制以移除恶意成员等等。

另一个独特的特征可以在DAOStack中找到,它是今天流行的一个DAO。它融入了“全息共识”,这是解决提案数量过多和决策要求带来挑战的一种方法。这种方法使得一个小组可以代表大多数人投票,与他们的愿望一致。DAOStack还采用了一种与投票权力不相关的单独指标称为“声誉”。这种方法旨在防止出现财富统治,根据他们的信念。

总的来说,DAO正在迅速发展,它们当前的设计差异很大。我们注意到了向提高安全性和可扩展性转变的显著转变。这种转变源于过去和现在的成员由于恶意行为者利用非限制性和简单结构而损失资产的实例。人们正努力提高投票、提案提交、成员资格和其他方面的安全性和健壮性,同时强调促进所有人的轻松和高效使用。然而,随着更多DAO的出现和对DAO行业的投资显著增加,它们有必要彼此区分开来。因此,我们观察到Aragon、Moloch和DAOStack具有不同的特征,每个特征都试图解决不同的挑战或创建新工具。可以肯定地说,DAOs还有很大的改进空间,我们预计未来的DAO设计将与现在大不相同。

\subsection{``一代币一票''与多类代币投票}

大多数DAO通过创建提案、将其提交给社区并进行社区投票来做出决策。社区成员可以将代币发送到DAO智能合约,智能合约然后计算票数并决定提案的批准或拒绝。每个代币代表一票,票数多数决定结果。然而,大多数DAO中对``多数''代币拥有者缺乏规定或限制,可能会造成重大的安全和安全风险。例如,如果联合创始人持有49.9%的总代币,并承诺在项目启动时将剩余的50.1%出售给社区,他们可以通过匿名购买0.2%的代币重新获得控制权。没有超出多数投票的规定,联合创始人可以无限制地铸造新代币或提取资金。

相比之下,现代公司持有股票遵循内部章程或公共规定,关于``后续公开发行(FPOs)''等流程。联合创始人和大股东必须向公众披露FPO信息,并且FPO受到总股数的限制。政府或监管机构,如美国证券交易委员会、中国证监会或香港证券及期货事务监察委员会,根据股市法律对FPO进行监管。在私人公司中,联合创始人和大股东可以通过称为``私人配售''的过程发行更多股份,但这不受股市法律的监管。然而,他们仍然需要向公众披露有关私人配售的信息。早期投资者面临严格的股份限制,如锁定期、归属期和他们可以出售的股份数量限制。联合创始人和大股东还必须披露有关这些限制的信息。此外,不同的股东和股份可能有附加的限制,如非稀释股份、反稀释股份和超级投票股份,这些都必须向公众披露。

在现代股份公司中,虽然重大问题需要所有人投票,但日常运营通常由高管和董事会审查和批准,而不是通过所有股东的普遍投票。相比之下,许多DAO今天完全依靠全员投票来批准社区提案,这可能既低效又耗时。DAOs通常设定投票门槛,例如51%的票数,如果达到门槛,则提案被批准。他们还设定投票持续时间,例如一天,如果持续时间满足,则提案被批准。一些DAO提供投票奖励,例如总代币的1%,分发给多数投票者。此外,为了防止连续的提案过多,DAOs通常设定创建提案的最低代币持有门槛,例如10%。因此,大多数代币持有者除非满足最低代币要求,否则不被允许创建提案。DAO提案机制可以设计为一个阻塞队列,代币持有者必须等待之前的提案关闭才能创建新的提案。这需要为DAOs设计一个多级层次结构,使得多类代币投票系统成为可能。

从现代股份公司中学到的另一个教训是``超级投票股''的概念。在联合股份公司中,超级投票股的投票权比普通股更高,为股东在公司决策过程中提供更大的影响力。超级投票股的所有者在股东大会上可能拥有更多的投票权或对特定决策拥有否决权。公司通常使用超级投票股来授予创始人、高管或关键利益相关者对公司方向的更大控制权。虽然这可以使领导层与长期利益保持一致,但它也可能造成利益冲突,并引发关于问责制和公平性的担忧。

在公司治理中使用超级投票股可能是有争议的,因为它给予少数个人对公司的不成比例影响。因此,法律和政策对超级投票股的使用进行了规定,以平衡不同利益相关者的利益。值得考虑将这一概念作为设计DAOs的潜在方式,尽管应谨慎地考虑使用超级投票股,同时考虑公司治理原则。

\subsection{公司章程和程序}

公司的章程包括规定其内部运营和管理的规则和规定。这些章程涵盖各种主题,包括组织结构、官员和董事责任、会议和投票程序以及股份发行和转让规则。

通常,董事会通过公司的章程,然后必须由股东批准。这些章程的目的是为公司的治理和决策过程建立一个清晰和一致的框架。它们定期被审查和更新,以与任何运营或法律变化保持一致。

在某些情况下,来自政府机构或行业机构的外部规定也可能适用于公司的章程。例如,上市公司必须确保遵守证券法律和其他相关规定。尽管如此,章程主要作为一个内部文件,以确保公司的顺利和有效运作。

在自我调节的公司中,章程侧重于定义禁止的行为,而不是允许的行为。它们提供了解决未来可能出现的问题的程序。

然而,在当前的DAO设计中,通常只有一种程序:多数投票。大多数DAO只对投票过程施加限制或规定,如投票门槛、持续时间、奖励和提案创建的最低代币要求。然而,没有章程概述禁止的行为或建立解决潜在问题的程序。因此,大股东可以轻易地通过提出损害小股东的行动来控制DAO,例如稀释他们的代币持有量,或执行“rug pull”提案以从DAO中提取资金。在这种情况下,DAO缺乏通过智能合约代码或社区监督的自我调节。随着决策完全依赖于多数投票,大股东可以操纵DAO并提出对他们有利但对他人不利的提案,导致“暴政多数”情景。

在健康、稳定和健全的商业实体中,如联合股份公司或非营利组织,章程对于定义禁止的行为和建立解决潜在未来问题的程序至关重要。股东、核心高管和董事会必须在加入或投资之前设定组织内的运营和权利界限。这使他们能够预防利益冲突并解决潜在问题。同样,在DAO的设计中,章程对于在其他代币所有者可以从他们那里购买代币之前对不同的代币所有者施加额外的限制至关重要。这种方法超出了简单的代币投票,因为仅依靠投票来解决所有提案不能解决所有问题,甚至可能通过通过“暴政多数”提案来创造进一步的问题。

\subsection{``代币经济''与``分红''}

对于DAO和其他加密项目(SocialFi、GameFi或其他协议),代币经济通常被用作吸引和留住投资者的手段,与分红相比。代币经济是指在加密货币系统内设计和实施代币的研究。它涉及创建一个代币,这是用于特定加密货币生态系统内交换的数字资产。代币在系统内代表价值,使得购买和销售商品和服务成为可能,以及激励期望的行为。代币的经济设计对于加密货币的整体成功至关重要,因为它决定了其随着时间的推移的效用和价值。一个设计良好的代币经济有助于确保加密货币的长期可行性和稳定性。

这引出了一个问题:DAO中的代币和传统公司中的股票是否相同?购买或赚取代币是否等同于投资股票?代币持有者是否被视为投资者或客户?代币是否可以作为投资目的的股票或出售的产品进行比较?

在美国SEC监管的真实世界公司中,没有公司仅仅为了向市场出售``股票''而存在,且没有规定就发行10到1000倍的股票是不可行的。``股票''与``产品''之间有明确的区别:``股票''代表公司的所有权,不仅包括资产和市场价值的所有权,还包括通过公司章程、合同和规定概述的特定程序参与公司决策过程的权利。相反,``产品''指的是生产和销售的商品或服务。它可以包括供销售的特定项目或向客户提供的整体产品。本质上,公司通过创建并销售商品或服务作为``产品''来产生收入,同时通过在一级或二级市场发行和销售新股来筹集资金,从而稀释现有股东的所有权。这种机制构成了传统股票市场的核心,推动股东、高管、员工和董事会成员合作创建有形价值,同时从真实商业活动中获利。

在大多数代币经济设计中,代币的精确定义往往缺失。许多DAO和加密项目利用代币来吸引和留住投资者,这些代币可以在中心化(CEX)或去中心化(DEX)交易所上交易并具有投票权。然而,在大多数代币经济设计中,代币并不等同于投资目的的股票。首先,在大多数代币经济算法中,核心智能合约持续地铸造新代币作为奖励给SocialFi用户、GameFi用户和DAO投票者。这种大量增加的代币供应导致价值稀释,降低了单个代币的重要性和所有权权重。其次,智能合约通常缺乏保障代币所有者权利的规定或程序,除了“一代币一票”原则,这通常是低效和耗时的。它无法防止联合创始人执行rug pulls,其中他们欺诈性地提取其他股东的资产。最后,正如在各种情况下观察到的,如果联合创始人可以通过“代币经济”方法通过铸造和销售代币筹集足够的资金,他们可能会停止实际的业务扩展,允许他们带着资金逃跑(也称为“rug pull”)并放弃项目。因此,代币不等同于投资目的的股票,项目的“业务”仅围绕“销售代币”,而不是创建和销售有形的商品或服务。因此,出于投资目的购买代币的代币所有者与股票投资者不同,尽管他们相信虚假的``代币经济''方法的幻觉,并期望随着项目增长代币价格升值,这通常产生相反的结果。

在健康和良好监管的股票市场中,大多数公司定期向其股东分配股息。超过80%在标普500指数中上市的公司定期支付股息。分红对于传统股票公司具有重要意义,因为它们允许公司与股东分享其利润。当公司产生利润时,它有多种选择来使用这笔钱。它可以将资金再投资于业务中以促进增长和扩张,偿还债务,或将部分利润以股息形式分配给股东。支付股息有助于公司吸引和留住寻求投资定期收入的投资者。此外,持续的股息支付表明财务稳定性和实力,增强了投资者对公司的信心。一些公司,特别是那些处于高增长行业或优先考虑将利润再投资以推动增长的公司,可能选择不支付股息。

公司选择不支付股息的几个原因。首先,公司可能没有多余的现金来分配给股东。如果公司正在大量投资其运营、研发或其他增长机会,它可能没有可用的现金进行股息支付。在这种情况下,公司可能保留利润以重新投资于业务并推动未来增长。

其次,公司可能认为支付股息不是为其股东创造价值的最有效方式。例如,一个快速增长的公司可能认为将利润再投资于业务为股东创造的价值大于分配股息。在这种情况下,公司专注于收入和盈利增长,期望这将导致更高的股价和更好的股东回报。

此外,一些公司选择放弃股息以保留资本配置决策的灵活性。通过保留利润而不是支付股息,公司可以利用其现金进行收购、战略投资或抓住其他机会。这为他们提供了对未来的更大控制,并有助于建立更强大和更有韧性的业务。

总之,“代币经济”不是设计加密项目或DAO的唯一或适当的方法。与“无限制铸造代币”相比,股息提供了一种更健康和更稳定的方法来稳定市场中的代币/股票价格。然而,股息在当前的加密项目中并不广泛使用。

\end{document}
