\documentclass[main.tex]{subfiles}
\begin{document}

\section{插件}

\subsection{插件的设计}

插件是DARC协议中的法律,DARC内的所有程序和操作都必须遵守所有插件施加的限制。对于单个插件,它遵循下面伪代码中概述的逻辑:

\begin{verbatim}
if plugin.condition:
    return plugin.returnType
\end{verbatim}

对于DARC协议,操作前插件和操作后插件之间的主要区别在于它们的返回类型。对于操作前插件,由于它们决定某个操作是否应该直接执行、直接拒绝或进入沙箱,因此它们有三种不同的返回类型作为它们的最终决定:

\begin{enumerate}
    \item \texttt{NO}。当操作前插件的条件被触发时,插件对该操作的决定是\texttt{NO}。这个决定表明插件认为操作违反了它的规则,因此在进入沙箱执行之前就被直接拒绝。

    \item \texttt{SANDBOX\_NEEDED}。当操作前插件的条件被触发时,插件对该操作的决定是\texttt{SANDBOX\_NEEDED}。这个决定表明插件无法确定操作应该被接受还是被拒绝。插件知道需要在沙箱中由操作后插件评估该操作,因此做出让操作进入沙箱以进行进一步评估的决定。

    \item \texttt{YES\_AND\_SKIP\_SANDBOX}。当操作前插件的条件被触发时,插件对该操作的决定是\texttt{YES\_AND\_SKIP\_SANDBOX}。这个决定表明插件已经确定操作应该被批准,并且不需要在沙箱中执行。因此,操作可以直接进行,无需经过沙箱。
\end{enumerate}



对于操作后插件,由于程序已在沙箱中执行并且可以开始投票,这些插件也可以有三种返回类型作为它们的最终决定:

\begin{enumerate}
    \item \texttt{NO}。当操作后插件的条件被触发时,插件对操作的决定是\texttt{NO}。这个决定表明插件认为操作违反了它的规则,应该被直接拒绝。

    \item \texttt{VOTING\_NEEDED}。当操作后插件的条件被触发时,插件对操作的决定是\texttt{VOTING\_NEEDED}。这个决定表明插件认为操作需要投票,并且操作需要根据该插件指定的投票规则初始化投票项。

    \item \texttt{YES}。当操作后插件的条件被触发时,插件对操作的决定是\texttt{YES}。这个决定表明插件根据它的规则认为操作应该被允许进行。
\end{enumerate}


每个插件都有一个条件节点数组,条件节点按顺序存储。根节点对应于索引0处的节点,即第一个节点。条件节点数组遵循以下原则:

\begin{enumerate}
    \item 每个节点的类型可以是布尔运算符或表达式;
    \item 对于布尔运算符,类型必须设置为AND、OR或NOT之一;
    \item 对于AND和OR运算符,必须在子节点列表中指定至少两个有效的子节点索引;
    \item 对于NOT运算符,必须在子节点列表中指定一个唯一的子节点索引;
    \item 对于表达式节点,必须设置与表达式一致的有效条件表达式参数;
    \item 对于表达式节点,其子节点列表的长度必须为0,意味着不允许有子节点。
\end{enumerate}

图 \ref{fig:condition-nodes} 是一个例子,说明了如何将一个条件表达式二叉树序列化为条件节点数组。

\begin{figure}
\centering
\includegraphics[width=1\linewidth]{plugin_condition_nodes.drawio.png}
\caption{\label{fig:condition-nodes}条件节点和表达式树}
\end{figure}

此外,插件需要设置两个参数:一个是``级别'',代表插件在整个插件系统中的优先级。对于同一个操作,判断系统遍历所有插件,可能至少有两个或更多插件被触发。在这种情况下,如果插件的级别不同,判断系统会将级别更高的插件视为最终决定。

另一个参数是``投票规则索引'',它指向投票规则数组中的一个特定索引。当插件的最终决定是\texttt{VOTING\_NEEDED}时,插件请求DARC协议使用投票规则索引指示的投票规则来初始化投票项。如果插件的返回类型不是\texttt{VOTING\_NEEDED},则会忽略投票规则索引。



\subsection{插件与判断系统}

对于DARC协议,判断系统需要进行两次评估:一次通过操作前插件,另一次在程序完全在沙箱中运行后,然后通过操作后插件进行评估。这样设计的原因是,如果没有沙箱并且仅依赖一组插件进行判断,预测程序的行为将变得非常困难。因此,无法保护DARC协议中特殊状态的修改。

例如,在一个DARC实例中,要求股东X永久持有15\%的投票权和10\%的分红权时,设计插件时无法预测铸造代币或销毁代币等操作执行后DARC实例的状态。这种不确定性给确保股东X永久拥有15\%的投票权和10\%的分红权带来了挑战。只有在沙箱中运行操作然后重新评估沙箱的状态,才能防止这种修改。

在另一个场景中,如果需要确保一个DARC实例在2035年1月1日之前永久保留10000个原生代币,只有在沙箱中执行这些操作后,才能通过操作后插件进行第二次评估,从而确保正确检测和防止支付分红或提取现金等操作。如果没有沙箱并且仅依赖插件,设计这样的机制将过于复杂。


如果只有操作后插件和沙箱而没有操作前插件,那将是昂贵且低效的。这是因为沙箱的运行成本非常高。它不仅需要在沙箱中完全运行程序,还涉及通过完全复制整个DARC实例的内部状态来初始化沙箱。这个过程会产生大量的Gas费用。

对于大多数可以在不需要在沙箱中运行的简单操作,直接在操作前插件中建立规则更为节省成本。例如,小股东和零售投资者之间的股份交易、客户进行日常交易、董事会成员执行常规支付操作以及员工为自己发放薪水和股票激励——这些众多的日常和高频活动可以定义为操作前插件。这种方法有助于为大多数日常操作节省Gas费用。



对于操作前插件,每当一个程序提交给DARC协议时,判断系统会顺序检查每个操作。对于每个操作,判断系统遍历每个操作前插件,并获得一个单一的判断结果。最后,它汇总所有操作的结果,以确定整个程序的最终结果。这个决定决定了程序是否需要在沙箱中运行(\texttt{SANDBOX\_NEEDED})、被直接拒绝(\texttt{NO})还是可以直接运行而不需要沙箱(\texttt{YES\_AND\_SKIP\_SANDBOX})。对于操作前判断,遵循以下规则:

\begin{enumerate}
    \item 如果任何操作被判断系统判定为\texttt{NO},整个程序将以\texttt{NO}的结果被拒绝。
    \item 如果没有任何操作被判断系统判定为\texttt{NO},并且至少有一个操作被判定为\texttt{SANDBOX\_NEEDED},整个程序需要在沙箱中运行,结果为\texttt{SANDBOX\_NEEDED}。
    \item 如果所有操作都被判断系统判定为\texttt{YES\_AND\_SKIP\_SANDBOX},整个程序将以\texttt{YES\_AND\_SKIP\_SANDBOX}的结果被批准,整个程序可以跳过沙箱。
\end{enumerate}




\begin{figure}
\centering
\includegraphics[width=1\linewidth]{judgement_plugin_levels_before_ops.drawio.png}
\caption{\label{fig:judgement-before-op}带有操作前插件的程序判断}
\end{figure}


图 \ref{fig:judgement-before-op} 说明了程序内的单个操作如何通过操作前插件被判断,得出判断结果,决定程序是否需要通过投票批准(\texttt{VOTING\_NEEDED})、直接拒绝(\texttt{NO})或可以直接进行(\texttt{YES})。对于操作后判断,遵循以下规则:

\begin{enumerate}
    \item 如果任何操作被判断系统判定为\texttt{NO},整个程序将以\texttt{NO}的结果被拒绝。
    \item 如果判断系统没有将任何操作判定为\texttt{NO},并且至少有一个操作被判定为\texttt{VOTING\_NEEDED},整个程序的最终判断为\texttt{VOTING\_NEEDED}。这个程序将被归类为待决程序,DARC协议必须启动投票系统来决定批准或拒绝。
    \item 如果所有操作都被判断系统判定为\texttt{YES},整个程序将以\texttt{YES}的结果被批准,整个程序可以直接执行。
\end{enumerate}


\begin{figure}
    \centering
    \includegraphics[width=1\linewidth]{judgement_plugin_levels_after_ops.drawio.png}
    \caption{\label{fig:judgement-after-op}带有操作后插件的程序判断}
\end{figure}

图 \ref{fig:judgement-after-op} 说明了程序内的单个操作如何通过操作后插件被判断,得出判断结果。


\end{document}
