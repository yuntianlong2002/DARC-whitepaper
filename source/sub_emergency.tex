\documentclass[main.tex]{subfiles}
\begin{document}
\section{紧急代理}

在DARC协议中,用户经常遇到各种问题,例如:

\begin{enumerate}
    \item 设置不正确的插件参数、触发条件或执行错误的代币操作,使得恢复变得不可能。
    \item 锁定DARC协议中的某些关键操作,导致协议内功能永久不可用。
    \item 人为争议导致组织无法运作,这些争议无法通过插件解决,但可能通过使用文件、文本、证据等手动诉讼来解决。
    \item 发现DARC协议的漏洞或面临攻击,需要紧急暂停和恢复。
    \item 解决可能出现的其他潜在技术问题或争议。
\end{enumerate}

在DARC协议中,用户可以为紧急情况指定一个或多个紧急代理。在不可预见的情况下,用户可以邀请紧急代理进行干预。紧急代理作为一个超级管理员,拥有在DARC协议内执行任何操作的权限,不受插件限制。这个角色对于解决DARC协议内紧急或未解决的问题至关重要。

\begin{enumerate}
    \item \texttt{addEmergency(emergencyAgentAddress)}:此命令用于通过提供紧急代理的地址来添加紧急代理。一旦添加,紧急代理获得超级管理员权限,允许他们在DARC协议中执行任何操作,不受插件的限制。

    \item \texttt{callEmergency(emergencyAgentAddress)}:此命令用于通过指定要调用的紧急代理的地址来调用紧急代理。调用后,紧急代理可以采取必要的紧急措施来处理不可预见的情况,并执行操作以确保DARC的正常运行。

    \item \texttt{endEmergency()}:此命令用于结束紧急状态。一旦紧急情况得到解决或处理,用户和紧急代理可以使用此命令结束紧急状态,并恢复DARC协议的正常操作。
\end{enumerate}

\end{document}
